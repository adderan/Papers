\documentclass{article}
\usepackage[utf8]{inputenc}

\usepackage[margin=1.0in]{geometry}
\begin{document}

\section{Research Goals}
Repetitive sequence makes up over half of the human genome, with up to 40\% being transposable elements, or transposons. The most common type of transposons are retrotransposons, which perpetuate themselves by being transcribed into RNA and then reverse-transcribed into DNA and inserted back into the genome. Examples of these transposable elements are Alu, a ~300bp retrotransposon that makes up about $11\%$ of the genome, and LINE1, a $6kb - 8kb$ retrotransposon that makes up about $15\%$. \cite{smit_origin_1996} LINE1 encodes reverse transcriptase itself, while Alu likely depends on harnessing an existing reverse transcriptase produced by LINE1. These transposons are interesting both as causes of disease, for example when Alu or LINE1 are inserted into functional genes, and as drivers of evolution, such as when LINE1 transposition causes structural rearrangement or gene duplication. \cite{kazazian_mobile_2004}

While repetitive elements like transposons are evolutionarily interesting, they also prohibitively increase the computational complexity of bioinformatics tools such as BLAST-like aligners, and result in redundant matches when querying databases. BLAST-like aligners use a “seed and extend” strategy to approximate Smith-Waterman alignment, in which exact kmer matches between the sequences are used to guess where high-scoring local alignments are likely to occur, preventing the full computation of the Smith-Waterman alignment matrix. Under normal conditions, BLAST’s computational complexity scales as $O(n)$ rather than $O(n^2)$ with the sequence length. However, with 56\% of the human genome containing repetitive elements, a significant portion of the global Smith-Waterman dynamic programming matrix could be computed, increasing the total complexity back to $O(n^2)$, which is prohibitive for whole genomes. Therefore, repetitive elements are often masked before performing pairwise alignment.

The most commonly-used computational tool for annotating and masking repeats is RepeatMasker, which checks the input sequence against a database of known repetitive sequences. One disadvantage of RepeatMasker is the dependency on existing repeat libraries, which may not be accurate for newly-sequenced genomes. Other tools such as REPuter are able to classify repeats de novo, by searching for exact kmer matches and attempting to extend a consensus repeat sequence that represents each repeat family. But representing repeats by their consensus sequence is sometimes inadequate when a repeat consists of multiple sub-repeats, each of which also can occur independently of the larger repeat. Pevzner et. al proposed addressing this by representing repeat families with an A-Bruijn graph, also known as breakpoint graph or “sequence graph,” which is a generalization of the de Bruijn graph that allows for non-exact copies of a repeat to be “glued” together in the same way that exactly identical kmers are collapsed into single nodes in a de Bruijn graph. Pevzner created the RepeatGluer algorithm for constructing an A-Bruijn graph from a set of pairwise alignments. \cite{pevzner_novo_2004}


Graph-based repeat classification techniques could be refined further using the ProgressiveCactus whole-genome multiple alignment framework. ProgressiveCactus progressively aligns a set of genomes according to a provided species tree, inferring the ancestor genome at each internal node in the tree based on the alignments between its descendants. A ProgressiveCactus alignment of a set of genomes could provide useful information for repeat classification. Even with the input genomes masked by RepeatMasker, such that alignments only occur between non-repetitive regions, one could detect the expansion or contraction of a family of repeats by checking for insertions or deletions in each genome relative to its direct ancestor. This allows a natural way to model sub-repeats, which are presumably a result of transposition events that duplicate sequence containing multiple copies of a repeat introduced earlier in the species tree. \cite{hickey_hal:_2013}

A further enhancement of this repeat classifier could involve creating a ProgressiveCactus alignment from unmasked input genomes rather than masked ones, in order to capture all alignments between repetitive regions. This would require overcoming the problem of creating an initial set of pairwise alignments for ProgressiveCactus to build a sequence graph from. However, it may be possible to only compute a representative sample of these pairwise alignments, and take advantage of transitivity in the sequence graph to infer all of the alignment relationships. Given a repeat family with N non-exact copies, there are N(N-1)/2 possible pairwise alignments between all the repeats. But a minimum of only N alignments is necessary if alignment relations are transitive, which they are in a sequence graph. Cactus represents an alignment as two or more “threads”, each of which is a series of “blocks” (gapless alignments) connected by adjacency edges, which represent gaps in the alignment. Therefore a repeat family with N copies would be represented as a series of blocks of degree N, and a set of N pairwise alignments between the repeats would be sufficient to construct this multiple alignment while preserving alignment information between every occurrence of the repeat and every other occurrence. The difficulty in constructing this transitive closure of all alignment relations is correctly choosing enough repeats from each repeat family such that the entire family is transitively aligned, but not so many that the pairwise alignment step scales quadratically. I think it should be possible to come up with a strategy for sampling the kmer matches that are to be extended into pairwise alignments, which should reduce the total number of alignments computed.

\section{Proposed Coursework}
I plan to take statistics, computer science, and population genetics and evolution classes that support my research goals. One possible class schedule would be:


Fall 2017: BME 200, AMS 203 - Statistics, CMPS 250 - Information Theory
Winter 2018: AMS 206B - Bayesian Statistics, BME 201 - Scientific Writing
Spring 2018: BME 232 - Evolutionary Genomics, CMPS 242 - Machine Learning

Requirements already satisfied: BME 205, BME 230, BME 80G

I believe this is a reasonable amount of work per quarter. Spring 2018 would be the heaviest workload because those are two project-based classes. Obviously I’m open to suggestions about this.

\bibliographystyle{unsrt}
\bibliography{cactus_repeats}

\end{document}
